\pagenumbering{arabic}
\section{绪论}

\subsection{研究背景}

随着互联网的高速发展,为了提供更好的互联网服务,网络基础设备的也在不断增加,这其中就有大量的路由器,交换机,服务器等设备需要一个良好的工作环境温度,来保证设备的稳定运行。当前,传统方式维护各设备的环境依旧采用人力方式,造成大量的人力浪费和重复性劳动,因此迫切需要将已有的工作自动化、智能化。

目前,温度控制类型的产品在行业内一般设计为在机器上操作,无法远程控制,不能很好满足精确控制,节能的要求。本系统既可以远程控制温度,又可以实时监控温度,火焰,烟雾,并对火灾等突发情况做出预警,可以很好满足运维人员维护监控的需求来维持通信机房稳定运行。我校(西安石油大学)的各设备间的温度控制依旧采用人力进行控制,且无法及时获取温度,湿度,烟雾等关键环境信息,带来了一定的空调电力浪费和安全的隐患。针对以上的通信机房运维巡检工作,使用新兴的物联网技术,设计并实现了通信机房温度监控和智能化控制,解决了传统运维管理方式的效率地下,工作繁琐,无法实时介入处理的问题。还有很多生活中的场景可以用本系统来完成,因此本系统前景广阔,很有研究的意义。

基于Arduino的机房分布式温度控制系统是一个用于多机房的温度监控调节系统,由笔者独立开发,现在已经投入使用,承担了机房温度监控和控制的任务,到目前为止,该系统运行良好。

\subsection{国内外研究现状}

由于本课题涉及了多个方面,所以需要从多个方面去调查研究目前国外的使用情况和应用方式。这其中又主要包含三个方面:分别为对数据中心的温度的控制、Arduino开发板的应用现状、物联网云平台的应用。

\subsubsection{国内研究现状}

对于数据中心方面的温度控制情况如下:

在使用IDC机房的过程中也不断出现急需解决的问题,最为明显的是机房巨大的耗电量及室内温度严重超标的问题。更甚者在一些地方的IDC机房出现了因空调制冷量不足,机房温度过高威胁机房设备运行安全等问题,况且IDC业务同样具有高能耗的特征。因此若想建立成功的IDC机房则需要解决好机房空调系统的合理运行,这一过程已成为现如今IDC机房建设以及使用过程中不可避免的环节\upcite{邹燕2016idc}。

目前国内对于机房温度控制方面,在数据中心采用的是常规冷水空调系统、单冷源+板式换热器空调系统和双冷源+板式换热器空调系统,这种情况要保持空调常年开机运行。这种情况只适合大型机房,机器集中放置,制冷与散热要求异常严格的企业机房。并不适用于分布式放置设备和要求节能减排的学校,因此,其温度控制方面无太大参考价值。




Arduino是一个开发各类设备,让你能充分感知和控制物理世界的生态系统。Arduino是一个基于一系列单片机电路板的开源物理计算平台。目前国内外。由于Arduino的开放性和便捷性,针对Arduino的使用非常流行。例如在GitHub\footnote{GitHub是一个使用Git进行版本控制的基于网络的托管服务,主要用于管理程序员软件代码和项目。}以及arduino.cc\footnote{Arduino创作者平台,具有大量的可参考项目。 网址:\url{https://create.arduino.cc/projecthub/} }网站上有大量的应用示例。但是其基本都是针对Arduino为核心,进行软硬件扩展,只适用于单系统,无法针对数据进行收集和汇总,无法满足分布式系统的监控、控制等要求。

对于进行数据收集和处理的国内物联网云平台,较为规范、开放的云平台,国内主要有OneNET\footnote{OneNET是由中国移动打造的PaaS物联网开放平台,}、贝壳物联\footnote{\url{https://www.bigiot.net/} 物联网云平台,通过互联网以对话、遥控器等形式与你的智能设备聊天、发送指令,查看实时数据,跟实际需求设置报警条件,通过APP、邮件、短信、微博、微信等方式通知用户。}。前者只适用于多设备接入,界面复杂,并未对特定场景有优化。后者较为符合要求,但是超过一定连接设备需要付费。都不能很好满足校园机房分布式温度监控的要求。

温度控制器属于信息技术的前沿尖端产品,尤其是温度控制器被广泛用于工农业生产、科学研究和生活等领域, 数量日渐上升。温度控制器是一种温度控制装置,它根据用户所需温度与设定温度之差值来控制加热器运作,从而达到改变用户所需温度的目的。近百年来, 温度控制器的发展大致经历了以下阶段:
(1) 模拟、集成机械式温度控制器;(2)电子式智能温度控制器。目前,国内外上新型温度控制器正从模拟式向数字式、电子式由集成化向智能化、网络化的方向发展。现今基于单片机的温度控制系统在生产、安全保护以及节约能源等方面发挥了着重要作用。近年来,国内基于单片机的温度控制系统在技术上得到迅速发展,性能不断完善,功能不断增强,适用范围也不断扩大,市场占有率逐年增长,进入21 世纪后,智能的温控系统正朝着高精度、多功能、总线标准化、高可靠性及安全性、开发虚拟温控器和网络温控器、研制单片测温控温系统等高科技的方向迅速发展。

\subsubsection{国外研究现状}


\subsection{研究需求与意义}

为了保证通信机房设备的良好运行,温度的监控和控制十分必要。例如,对于通信机房中的交换机,如果温度偏高,就会导致机器散热堆积,使得集成电路和二极管等元器件形成结晶,热堆积严重时甚至会烧毁设备,相对的,温度过低时,会导致绝缘材料变脆,以及冷凝结露。这种情况下导致设备短路,也会对设备的寿命和可靠性造成影响。

基于对以上需求的分析,为了保证对于通信机房的温度控制和监控,基于Arduino的分布式温度控制系统有着明显的现实需求和意义。

本项目通过物联网技术进行数据共享和更新,具有分布式便于扩展,方便部署的特点,解决了线下运维巡检方面的重复性工作,保证了对于机房的温度精细控制。提供烟雾自动报警,无论从自动化,环保,安全角度考虑。这套系统对于机房管理人员都将提供更便捷,智能的工作方式。


\subsection{主要工作和设计}

本系统是对环境温度进行实时监测和控制,设计开发的分布式温度监测控制系统实现了基本的温度、湿度、烟雾的监控功能以及火灾报警功能:各个节点的Arduno通过传感器模块,采集到环境信息后,进行判断:当环境温度低于设定的下限温度时,通过红外模块,切换空调的模式为自动。当环境温度高于设定的上限时,通过红外模块,启动空调的制冷模式,并通过W5100以太网扩展模块以MQTT\footnote{MQTT 协议 是基于发布/订阅模式的物联网通信协议,凭借简单易实现、支持 QoS、报文小等特点,占据了物联网协议的半壁江山。}协议,发送相关的温度、湿度、烟雾到指定服务器。服务端采用Linux作为服务器,安装MQTT服务端接受数据,采用Python的Flask模块快速构建WEB应用程序,其负责对MQTT服务端的数据解析和数据分析:当烟雾值超过预设值时,自动发送邮件报警到指定邮箱。通过这些共同构建完成了基于Arduino的分布式温度控制系统。

\newpage 
实现的功能如下:

1. 监控点口管理功能。 自动发现Arduino监控节点功能,提供可视化的个管理与配置功能。配置监控节点的地址位置、IP地址阈值设置、管理人员等。

2. 关键指标的统计展示功能。 根据监控节点采集的温度、湿度、PM2.5、烟感等信息,结合监控节点的部署位置、指标阈值等。提供不同维度的信息展示功能。 

3. 温度控制功能。 根据采集到的的实时,提供通过红外传感器控制空调的温度的功能。 

4. 告警功能 设置系统级的告警阈值和单个节点的告警阈值,其中单个节点的阈值优先级高于系统级别; 系统级别的告警接收人和单个节点的告警接收人,单个节点告警接收与系统级别并存; 提供邮件等方式告警。

\subsection{本章小结}
本章主要论述基于目前国内外对于温度控制,Arduino,物联网平台的研究,通过对该方向的深入了解,确立合适的基于Arduino的分布式温度控制系统的技术选型以及整体架构设计。对论文中主要涉及的工作进行立项,确定基于Arduino的分布式温度控制系统的实现的具体功能和模块设计。