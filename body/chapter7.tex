% \fancypagestyle{summaryStyle}{
% 	\fancyhf{}
% 	\fancyhead[C]{\zihao{5} \songti 7\hspace{1em}结论与总结}
% }
% \thispagestyle{summaryStyle} 
\section{结论与总结}


依照基于Arduino的分布式温控系统的计划任务书的需求,首先进行系统的需求分析,之后进行整体设计,确定了Arduino客户端的程序流程和基于WEB温控系统的程序流程。论证了各项前沿的开发技术和开发方式,确定了整体项目框架以及数据库设计。采用了Arduino作为各节点信息采集的设备。其低廉的成本节省了大量的经费。Python的Flask作为开发WEB应用程序的框架,数据库采用了高效的MySQL。以及Linux作为操作系统。共同构成了完整的基于Arduino的分布式温度控制系统。

本系统基本完成了预期的目标:具有自动化的温度控制模块、远程可视化的环境温度、湿度、烟雾的监控模块,以及烟雾报警模块。基本满足了校园中通信机房智能化控制空调的能力。

本次系统开发离不开前期周密的技术调研,以及大量开源项目的使用。这些都很大程度的节省了开发代码的时间,并对该项目提供了很多方面的支持。

本次系统设计开发中,通过学习大量前沿项目经验,如ORM的数据库操控方式,RESTful的web架构,深入了解了最新的开发思想,开拓了编程思路。如ORM使得不必去书写复杂冗长的SQL命令语句,通过SQL的对象实例化,以编程的思路去控制数据库,这样减少不必要的数据库往返次数,增加了代码的一致性,提高了代码的复用性,使得编程效率有了很大的提高。


