\fancypagestyle{abstractStyle}{
	\fancyhf{}
	\fancyhead[C]{\zihao{5} \songti 摘\hspace{1em}要}
}
\thispagestyle{abstractStyle} 

\section*{\zihao{3} \centering 基于Arduino分布式温控系统设计与实现}
%\addcontentsline{toc}{section}{摘要}  % 摘要是否包括进目录中

\begin{center}
	\textbf{\zihao{-3} 摘\hspace{1em}要}
\end{center}
\vspace{2em}

随着物联网的快速发展,进入了万物互联的时代,越来越多的设备可以进行联网控制,通过网络进行数据收集以及对设备的远程控制。当下大量电子设备的使用,由于电子设备本身的温度的敏感性,对于环境温度的监控和控制也尤为重要。

随着越来越多的机房和交换机,路由器,服务器的部署,其中的温度监控和控制也越来越重要。有资料表明\footnote{热力学中最重要的公式——阿累尼乌斯方程,它是用来描述化学物质反应速率随温度变化关系的经验公式。阿累尼乌斯(Arrhenius)方程 $\ k=Ae^{{{-E_{a}}/{RT}}}$ 可以用来计算电容寿命,电容工作每下降10度,其寿命增加一倍,反过来也就是电容温度升高10度,电容寿命减小一倍}:环境温度每提高\SI{10}{\degreeCelsius},元器件寿命约降低30\%-50\%,影响小的也基本都在10\%以上,设备更持久的运行的需求,对于机房的温度就需要得到进一步的控制,这样才能增长使用的寿命。相对于传统的人员运维和监控,浪费大量的人力和财力资源,因此,迫切需要自动化可远程监控的管理平台,来把巡检和温度操控这种重复性劳动改革为自动化处理。

根据基于Arduino分布式温控系统设计与实现的需求,首先将功能抽象化,确定了采用C/S的结构模式。客户端采用Arduino开发板和传感器用来采集数据和控制红外发射温度控制信号,服务端采用LNMP (Linux+Nginx+Mysql+Python)进行远程信息的收集和监控以及警报。其中,采用了Linux为操作系统,Nginx为Web Server,Mysql对采集信息进行数据存储,Python的flask框架作为Web应用程序框架共同构成了服务端的温度监控报警系统。客户端与服务端之间由MQTT协议负责数据通信。

通过对于该系统的实际应用,由Arduino自动控制空调温度,平台监控温湿度、烟雾信息,既可保证机房内温度的均衡和稳定,又起到了节能减排的效果。并且增加了烟雾报警系统,对于火灾烟雾有了警报,方便运维人员更快速的处理突发情况,在温度控制方面减少了人工干预和处理,极大的解放了人力成本。


\vspace{1em}
\noindent
\textbf{关键词:}控制系统;分布式温度控制系统;Web开发;Arduino;Python


\clearpage
\fancypagestyle{abstractStyleEn}{
	\fancyhf{}
	\fancyhead[C]{\zihao{5} ABSTRACT}
}
\thispagestyle{abstractStyleEn} 

\section*{\songti\zihao{3} \centering \textbf{Design and Implementation of Distributed Temperature Control System based on Arduino}}
%\addcontentsline{toc}{section}{Abstract}  % 英文摘要是否包括进入目录

\begin{center}
	\textbf{\zihao{-3}ABSTRACT}
\end{center}
\vspace{2em}

With the rapid development of the Internet of Things and entering the era of the Internet of Everything, more and more devices can be controlled by the Internet, data collection and remote control of devices through the network. At present, the use of a large number of electronic devices is particularly important for monitoring and controlling the ambient temperature due to the temperature sensitivity of the electronic devices themselves.

Some data show that: every time the ambient temperature increases by \SI{10}{\degreeCelsius}, the life of the components is reduced by 30\%-50\%, and the impact is basically more than 10\%. The equipment needs more durable operation, and the temperature of the computer room needs to be further improved. Control, so as to increase the life span of use. Compared with traditional personnel operation and maintenance and monitoring, a lot of human and financial resources are wasted. Therefore, there is an urgent need for an automated management platform that can be remotely monitored to reform the repetitive labor of inspection and temperature alignment into automated processing.

According to the requirements of the design and implementation of the Arduino-based distributed temperature control system, the function is abstracted first, and the C/S structure mode is determined to be adopted. The client uses Arduino development boards and sensors to collect data and control infrared emission temperature control signals, and the server uses LNMP (Linux + Nginx + Mysql + Python) for remote information collection and monitoring and alarms. The web application framework together constitutes the temperature monitoring and alarm system on the server side. The MQTT protocol is responsible for data communication between the client and the server.

Through the actual application of the system, the Arduino automatically controls the temperature of the air conditioner, and the platform monitors the temperature, humidity, and smoke information, which can not only ensure the balance and stability of the internal temperature of the computer room, but also save energy. In addition, a smoke alarm system has been added to provide an alarm for fires, which facilitates faster handling by operation and maintenance personnel, reduces manual intervention and processing in terms of temperature control, and greatly liberates labor costs.

\vspace{1em}
\noindent 
\textbf{Keywords: }Control system; Distributed temperature control system; Web development; Arduino; Python

